% PROPOSITIONAL LOGIC + NATURAL DEDUCTION
% ==================================================
\makecn{PL}
\makecn{PLND}
\makecn{prop}

\newcommand\dedc{\cn{\mathnormal{\Vdash}}}
\newcommand\ded{\Vdash}

\newcommand\true{\cn{true}}

\newcommand\andc{\cn{\mathnormal{\wedge}}}

\newcommand\orc{\cn{\mathnormal{\vee}}}
\newcommand\orElimc{\cn{\orc E}}
\newcommand\orElim{\midflowed{\orElimc}}
\newcommand\orIntroLc{\cn{\orc IL}}
\newcommand\orIntroRc{\cn{\orc IR}}
\newcommand\orIntroL{\leftflowed{\orIntroLc}}
\newcommand\orIntroR{\leftflowed{\orIntroRc}}

% to typeset negation, use \neg (which is the same as \lnot as per LaTeX, see https://tex.stackexchange.com/questions/430677/what-is-the-difference-between-lnot-and-neg)

\newcommand\negc{\cn{\mathnormal{\lnot}}}
\NewDocumentCommand\negIntro{mo}{%
	\leftflowed{\cn{\negc I}_{#1\IfValueT{#2}{\colon #2}}}%
}
\newcommand\negElim{\midflowedcn{\negc E}}

\newcommand\impl{\Rightarrow}
\newcommand\implc{\mathnormal{\Rightarrow}}
\newcommand\biimpl{\Leftrightarrow}
\newcommand\biimplc{\mathnormal{\Leftrightarrow}}
\newcommand\biimplIntro{\leftflowedcn{\biimplc I}}
% elimination of the forward implication
\newcommand\biimplElimFwd{\midflowedcn{\biimplc \overset{{}_{\rightarrow}}{E}}}
% elimination of the backwrad implication
\newcommand\biimplElimBwd{\midflowedcn{\biimplc \overset{{}_{\leftarrow}}{E}}}

% latex already defines \equiv, so use \equivs (s for "symbol")
% deprecated, use \biimpl
\newcommand\equivs{\biimpl}
% deprecated, use \biimplc
\newcommand\equivc{\biimplc}
% deprecated, use \biimplIntro
\newcommand\equivIntro{\equivc_{\cn{I}}}
% deprecated, use \biimplElimL
\newcommand\equivElimL{\biimplElimL}
% deprecated, use \biimplElimR
\newcommand\equivElimR{\biimplElimR}

\newcommand\falsum{\cn{\mathnormal{\bot}}}
\newcommand\falsumIntro{\falsum_{\cn{I}}}

% Double-stroked not symbols
%
% Source: https://tex.stackexchange.com/a/41141/38074
% Author: egreg <https://tex.stackexchange.com/users/4427/egreg>
% License: CC BY-SA 3.0 <https://creativecommons.org/licenses/by-sa/3.0/>
\newcommand{\blnot}{\mathord{\mathpalette\xblnot\relax}}
\newcommand{\xblnot}[2]{
	\sbox2{$#1\lnot$}\vrule height 1.1\ht2 width0pt \ooalign{%
		\hphantom{$#1-$}\relax\cr
		\noalign{\vskip-.2\ht2}
		\hfil$#1\lnot$\hfil\cr
		\noalign{\vskip.4\ht2}
		\hfil$#1\lnot$\hfil\cr
		\noalign{\vskip-.2\ht2}}%
}
% END: Double-stroked not symbols
\newcommand\doublenegc{\cn{\mathnormal{\blnot}}}
\newcommand\doubleNegIntro{\doublenegc_{\cn{I}}}
% ==================================================


% UNTYPED AND TYPED (SORTED) FIRST-ORDER LOGIC
% ==================================================
\makecn{FOL}
% sorted (aka typed) FOL; polymorphic FOL; dependent FOL; polymorphic+dependent FOL
\makecn{SFOL}\makecn{TFOL}\makecn{PFOL}\makecn{DFOL}\makecn{PDFOL}
\makecn{tp}
\makecn{ind}
\newcommand\forallc{\cn{\mathnormal{\forall}}}

% use \doteq for \eq relation
% and \eqc for raw symbol
\newcommand\eqc{\cn{\mathnormal{\doteq}}}

% meta-level equality on SFOL types
\newcommand\tpeq{\mathrel{\overset{\ast}{=}}}
\newcommand\tpeqc{\mathnormal{\overset{\ast}{=}}}

\RequirePackage{esvect}
% transport value #1 along type equality #2
\NewDocumentCommand\tptrans{mm}{\vv{#1}^{#2}}

% forall (weird name to not clash with latex's \forall)
\NewDocumentCommand\fall{m o}{\quantifier{\forall\,}{#1}[#2]}

% exists (weird name to not clash with latex's \exists)
\NewDocumentCommand\xists{m o}{\quantifier{\exists\,}{#1}[#2]}

\NewDocumentCommand\of{}{\mathbin{::}}
\NewDocumentCommand\ofc{}{\mathnormal{\of}}
%\newcommand{\denseOfTermType}[2]{#1\kern -0.1em::\kern -0.1em#2}

%\newcommand{\spacedtmS}[2]{\tmS{\kern 0.05em#1\kern 0.06em#2}}

% Typeset "\forall #1:#2, ..., #3:#4." with good spacing and dense ldots
% The types #2 and #4 are optional (and thus given via square brackets)
% Examples: `\flexForall{s}[S]{t}[T]`
%           `\flexForall{s}{t}`
\NewDocumentCommand\flexForall{m o m o}{\ensuremath{%
		\forall\, #1\IfValueT{#2}{\kern -0.1em\colon\kern -0.08em#2}%
		,\denseLdots,%
		#3\IfValueT{#4}{\kern -0.1em\colon\kern -0.08em#4}%
		.\hspace{0.6em}
}}

% Typeset "\forall #1:#2, ..., #3:#4." with dense spacing and dense ldots
%\newcommand{\denselexaryForall}[4]{\ensuremath{\forall\, #1\colon#2,\denseLdots,#3\colon#4.\hspace{0.6em}}}

% untyped function symbol type
\NewDocumentCommand\ufunSymType{}{\flexHomogeneousFunType{\term}{\term}}
\NewDocumentCommand\funSymType{mmm}{\flexFunType{\tm{#1}}{\tm{#2}}{\tm{#3}}}
% untyped predicate symbol type
\NewDocumentCommand\upredSymType{}{\flexFunType{\term}{\term}{\prop}}
\NewDocumentCommand\predSymType{mm}{\flexFunType{\tm{#1}}{\tm{#2}}{\prop}}

% Typeset dense "\colon\ded"
\newcommand\colonded{\colon\kern -0.25em\ded}
%% END: sorted first-order logic %%

%% NATURAL DEDUCTION FOR SORTED FIRST-ORDER LOGIC %%
\newcommand\forallElimc{\cn{\mathnormal{\forall}E}}
\newcommand\forallElim{\midflowedcn{\mathnormal{\forall}E}}

% natural deduction kern preceeding other natural deduction macros below
\newcommand\ndleftkern{\kern0.3em}
% natural deduction following other natural deduction macros below
\newcommand\ndrightkern{\kern0.3em}

% "flowed"
\newcommand{\leftflowed}[1]{#1\ndrightkern}
\newcommand{\leftflowedcn}[1]{\leftflowed{\cn{#1}}}
\newcommand{\midflowed}[1]{\mathbin{\ndleftkern#1\ndrightkern}}
\newcommand{\midflowedcn}[1]{\midflowed{\cn{#1}}}
\newcommand{\rightflowed}[1]{\ndleftkern#1}
\newcommand{\rightflowedcn}[1]{\rightflowed{\cn{#1}}}

\NewDocumentCommand\forallIntro{mo}{%
	\leftflowed{\cn{\mathnormal{\forall}I}_{#1\IfValueT{#2}{#2}}}%
}
\NewDocumentCommand\forallIntroc{}{\cn{\mathnormal{\forall}I}}%
\NewDocumentCommand\existsElim{mm}{%
	\midflowed{\cn{\mathnormal{\exists}E}_{#1,#2}}%
}
\NewDocumentCommand\existsElimc{}{%
	\cn{\mathnormal{\exists}E}%
}

\newcommand\frefl{\leftflowedcn{refl}}

\makecn{refl}
\makecn{symm}
\makecn{trans}

\newcommand\fexistsIntro{\leftflowedcn{existsI}}
\newcommand\existsIntro{\leftflowedcn{\exists I}}
\NewDocumentCommand\existsIntroc{}{\cn{\mathnormal{\exists}I}}%
\newcommand\fexistsElim{\midflowedcn{existsE}}

\newcommand\fimplIntro{\leftflowedcn{implI}}
\newcommand\fimplElim{\midflowedcn{implE}}

\newcommand\fandIntro{\leftflowedcn{andI}}
\newcommand\fandElimL{\rightflowedcn{andEL}}
\newcommand\fandElimR{\rightflowedcn{andER}}

\newcommand\forIntroL{\leftflowedcn{orIL}}
\newcommand\forIntroR{\leftflowedcn{orIR}}
\newcommand\forElim{\midflowedcn{orE}}

\newcommand\fforallIntro{\leftflowedcn{forallI}}
\newcommand\fforallElim{\midflowedcn{forallE}}

\newcommand\fdepimpl{\midflowedcn{depImpl}}
\newcommand\fdepand{\midflowedcn{depAnd}}

% use in \begin{align*} instead of \\ to issue a bigger break useful to separate things (e.g. inclusions and actual theory body)
\newcommand\alignskip{\\[0.6em]}
% If you want to typeset, e.g., "... forallE blub", use `\dotsBeforeFlow \fforallElim blub`.
% Do _not_ employ `\denseLdots \fforallElim blub` as this will miss suitable space before \denseLdots.
\newcommand\dotsBeforeFlow{\ndleftkern\denseLdots}
%\makecn{cong} clashes with latex's \cong (isomorphism symbol)
%\newcommand\fcong{\midflowedcn{cong}}

% typeset variable name "pf" (and variants) for LF variables representing proofs (in natural deduction calculi)
% \pf* typesets "pf'"
% \pf[x] typesets "{pf^{x}}" (thus it enables you to use `\lam{\pf[x]^p \pf[x]^s}`)
\NewDocumentCommand\pf{so}{%
	\IfValueTF{#2}{%
		{\mathit{pf^{#2}}}%
	}{%
		\IfBooleanTF{#1}{{\mathit{pf'}}}{\mathit{pf}}%
	}%
}

% for constant identifiers of axiom symbols
\newcommand\ax{\mathit{ax}}
%% END: natural deduction for sorted first-order logic %%
% ==================================================

% (DEPENDENT) CONJUNCTION
% ==================================================
\NewDocumentCommand\depandc{}{\andc_{-}}

\NewDocumentCommand\depand{mo}{%
	\ndleftkern\wedge_{#1\IfValueT{#2}{\colon #2}}\ndrightkern%
}
% alias for \depand
\NewDocumentCommand\depwedge{mo}{\depand{#1}[#2]}

\newcommand\andIntro{\leftflowed{\andIntroc}}
\newcommand\andIntroc{\cn{\mathnormal{\wedge}I}}
\NewDocumentCommand\depandIntro{}{%
	\leftflowedcn{\mathnormal{\wedge}I}%
}

\newcommand\andElimLc{\cn{\mathnormal{\wedge}EL}}
\newcommand\andElimL{\rightflowed{\andElimLc}}

\newcommand\andElimRc{\cn{\mathnormal{\wedge}ER}}
\newcommand\andElimR{\rightflowed{\andElimRc}}

\NewDocumentCommand\depandElimL{}{\rightflowedcn{\mathnormal{\wedge}EL}}
\NewDocumentCommand\depandElimR{}{\rightflowedcn{\mathnormal{\wedge}ER}}
% ==================================================

% (DEPENDENT) IMPLICATION
% ==================================================
\NewDocumentCommand\depimplc{}{\implc_{-}}
\NewDocumentCommand\depimpl{mo}{\midflowed{\implc_{#1\IfValueT{#2}{\colon #2}}}}

\NewDocumentCommand\implIntro{mo}{%
	\leftflowed{\cn{\implc I}_{#1\IfValueT{#2}{\colon #2}}}%
}
\NewDocumentCommand\depimplIntro{mo}{%
	\leftflowed{\cn{\implc I}_{#1\IfValueT{#2}{\colon #2}}}%
}

\NewDocumentCommand\implElim{}{%
	\midflowedcn{\implc E}%
}
\NewDocumentCommand\depimplElim{}{%
	\midflowedcn{\implc E}%
}
% ==================================================

% SORT OPERATORS
% ==================================================

% for predicate subtypes
% \usepackage[llbrace,rrbrace]{stmaryrd} --> option clash?
% ideally, we would use llbrace and rrbrace from stmaryrd
\NewDocumentCommand\subtype{m m}{%
	\lVert #1 \kern-0.35em\mid\kern-0.35em #2 \rVert%
}
\NewDocumentCommand\downcast{m m}{#1 \kern-0.2em\downarrow\kern-0.2em #2}
\NewDocumentCommand\upcast{m}{#1\kern-0.1em\uparrow}
\NewDocumentCommand\proofcast{m}{#1 ?}

% for quotient types
\NewDocumentCommand\quottp{mm}{%
	#1 / #2
}
% ==================================================