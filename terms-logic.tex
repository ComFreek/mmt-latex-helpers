% PROPOSITIONAL LOGIC + NATURAL DEDUCTION
% ==================================================
\makecn{PL}
\makecn{PLND}
\makecn{prop}

\newcommand\dedc{\cn{\mathnormal{\Vdash}}}
\newcommand\ded{\Vdash}

\newcommand\andc{\cn{\mathnormal{\wedge}}}
\newcommand\andIntro{\andc_{\cn{I}}}
\newcommand{\andElimIdx}[1]{\andc_{\cn{E#1}}}
\newcommand\andElimL{\andElimIdx{L}}
\newcommand\andElimR{\andElimIdx{R}}

\newcommand\orc{\cn{\mathnormal{\vee}}}
\newcommand\orElim{\orc_{\cn{E}}}
\newcommand\orIntroL{\orc_{\cn{IL}}}
\newcommand\orIntroR{\orc_{\cn{IR}}}

\newcommand\negc{\cn{\mathnormal{\lnot}}}
\newcommand\negIntro{\negc_{\cn{I}}}
\newcommand\negElim{\negc_{\cn{E}}}

\newcommand\equivs{\Leftrightarrow}% latex already defines \equiv, so use \equivs (s for "symbol")
\newcommand\equivc{\cn{\mathnormal{\Leftrightarrow}}}
\newcommand\equivIntro{\equivc_{\cn{I}}}
\newcommand\equivElimL{\equivc_{\cn{EL}}}
\newcommand\equivElimR{\equivc_{\cn{ER}}}

\newcommand\falsum{\cn{\mathnormal{\bot}}}
\newcommand\falsumIntro{\falsum_{\cn{I}}}

% Double-stroked not symbols
%
% Source: https://tex.stackexchange.com/a/41141/38074
% Author: egreg <https://tex.stackexchange.com/users/4427/egreg>
% License: CC BY-SA 3.0 <https://creativecommons.org/licenses/by-sa/3.0/>
\newcommand{\blnot}{\mathord{\mathpalette\xblnot\relax}}
\newcommand{\xblnot}[2]{
	\sbox2{$#1\lnot$}\vrule height 1.1\ht2 width0pt \ooalign{%
		\hphantom{$#1-$}\relax\cr
		\noalign{\vskip-.2\ht2}
		\hfil$#1\lnot$\hfil\cr
		\noalign{\vskip.4\ht2}
		\hfil$#1\lnot$\hfil\cr
		\noalign{\vskip-.2\ht2}}%
}
% END: Double-stroked not symbols
\newcommand\doublenegc{\cn{\mathnormal{\blnot}}}
\newcommand\doubleNegIntro{\doublenegc_{\cn{I}}}
% ==================================================


% UNTYPED AND TYPED (SORTED) FIRST-ORDER LOGIC
% ==================================================
\makecn{FOL}
% sorted (aka typed) FOL; polymorphic FOL; dependent FOL; polymorphic+dependent FOL
\makecn{SFOL}\makecn{TFOL}\makecn{PFOL}\makecn{DFOL}\makecn{PDFOL}
\makecn{tp}
\makecn{ind}
\newcommand\forallc{\cn{\mathnormal{\forall}}}
\newcommand\eqc{\cn{\mathnormal{\doteq}}}

% forall (weird name to not clash with latex's \forall)
\NewDocumentCommand\fall{m o}{\quantifier{\forall}{#1}[#2]}

% exists (weird name to not clash with latex's \exists)
\NewDocumentCommand\xists{m o}{\quantifier{\forall}{#1}[#2]}


%\newcommand{\of}[2]{#1\,::\,#2}
%\newcommand{\denseOfTermType}[2]{#1\kern -0.1em::\kern -0.1em#2}

%\newcommand{\spacedtmS}[2]{\tmS{\kern 0.05em#1\kern 0.06em#2}}

% Typeset "\forall #1:#2, ..., #3:#4." with good spacing and dense ldots
% The types #2 and #4 are optional (and thus given via square brackets)
% Examples: `\flexForall{s}[S]{t}[T]`
%           `\flexForall{s}{t}`
\NewDocumentCommand\flexForall{m o m o}{\ensuremath{%
		\forall\, #1\IfValueT{#2}{\kern -0.1em\colon\kern -0.08em#2}%
		,\denseLdots,%
		#3\IfValueT{#4}{\kern -0.1em\colon\kern -0.08em#4}%
		.\hspace{0.6em}
}}

% Typeset "\forall #1:#2, ..., #3:#4." with dense spacing and dense ldots
%\newcommand{\denselexaryForall}[4]{\ensuremath{\forall\, #1\colon#2,\denseLdots,#3\colon#4.\hspace{0.6em}}}

\NewDocumentCommand\funSymType{mmm}{\flexFunType{\tm{#1}}{\tm{#2}}{\tm{#3}}}
\NewDocumentCommand\predSymType{mm}{\flexFunType{\tm{#1}}{\tm{#2}}{\prop}}

% Typeset dense "\colon\ded"
\newcommand\colonded{\colon\kern -0.25em\ded}
%% END: sorted first-order logic %%

%% NATURAL DEDUCTION FOR SORTED FIRST-ORDER LOGIC %%
\newcommand\forallIntro{\cn{\mathnormal{\forall}I}}
\newcommand\forallElim{\cn{\mathnormal{\forall}E}}

% "flowed"
\newcommand{\leftflowedcn}[1]{\cn{#1}\kern0.3em}
\newcommand{\midflowedcn}[1]{\mathbin{\kern0.3em\cn{#1}\kern0.3em}}
\newcommand{\rightflowedcn}[1]{\kern0.3em\cn{#1}}

\newcommand\frefl{\leftflowedcn{refl}}
\makecn{refl}

\newcommand\fexistsIntro{\leftflowedcn{existsI}}
\newcommand\fexistsElim{\midflowedcn{existsE}}

\newcommand\fimplIntro{\leftflowedcn{implI}}
\newcommand\fimplElim{\midflowedcn{implE}}

\newcommand\fandIntro{\leftflowedcn{andI}}
\newcommand\fandElimL{\rightflowedcn{andEL}}
\newcommand\fandElimR{\rightflowedcn{andER}}

\newcommand\forIntroL{\leftflowedcn{orIL}}
\newcommand\forIntroR{\leftflowedcn{orIR}}
\newcommand\forElim{\midflowedcn{orE}}

\newcommand\fforallIntro{\leftflowedcn{forallI}}
\newcommand\fforallElim{\midflowedcn{forallE}}

\newcommand\fdepimpl{\midflowedcn{depImpl}}
\newcommand\fdepand{\midflowedcn{depAnd}}

% use in \begin{align*} instead of \\ to issue a bigger break useful to separate things (e.g. inclusions and actual theory body)
\newcommand\alignskip{\\[0.6em]}
% If you want to typeset, e.g., "... forallE blub", use `\dotsBeforeFlow \fforallElim blub`.
% Do _not_ employ `\denseLdots \fforallElim blub` as this will miss suitable space before \denseLdots.
\newcommand\dotsBeforeFlow{\kern0.3em\denseLdots}
\makecn{cong}
\newcommand\fcong{\midflowedcn{cong}}

% typeset variable name "pf" for LF variables representing proofs (in natural deduction calculi)
\newcommand\pf{\mathit{pf}}

% for constant identifiers of axiom symbols
\newcommand\ax{\mathit{ax}}
%% END: natural deduction for sorted first-order logic %%

% ==================================================

