\RequirePackage{amsmath,xparse,xpatch,xspace,xstring,listofitems}

%%% MACROS FOR TYPESETTING MMT TERMS %%%
\providecommand{\cn}[1]{\ensuremath{\mathtt{#1}}}
\newcommand{\makecn}[1]{\expandafter\def\csname#1\endcsname{\cn{#1}\xspace}}

% GENERAL
% ==================================================
\newcommand\denseTo{\kern -0.15em\to\kern -0.15em}

% Command \myDots (renamed to \denseLdots here) copied and adapted from <https://tex.stackexchange.com/a/200849>
% User: LaRiFaRi <https://tex.stackexchange.com/users/32245/larifari>
% License: CC BY-SA 3.0 <https://creativecommons.org/licenses/by-sa/3.0/>
\newcommand{\denseLdots}{\ifmmode\mathinner{\kern-0.06em\ldotp\kern-0.2em\ldotp\kern-0.2em\ldotp\kern-0.06em}\else.\kern-0.13em.\kern-0.13em.\fi}

% \quantifier{q}{x}      typesets `q x.\ `
% \quantifier{q}{x}[T]   typesets `q x: T.\ `
% \quantifier{q}{x y}    typesets `q x y.\ `
% \quantifier{q}{x y}[T] typesets `q x y: T.\ `
%
% see \lam, \tpi, \lampi
%
% If you want to use macros as bound variable names, be sure to end them via {} or some super/subscript, e.g.
%  with `\newcommand\hatx{\hat{x}}
%  instead of `\quantifier{q}{\hatx \hatx^y \hatx_z}` you have to write `\quantifier{q}{\hatx{} \hatx^y \hatx_z}`
%  holds for all derived macros, too (esp. \lam, \tpi, ...)
\NewDocumentCommand\quantifier{m m o}{%
	#1
	\setsepchar{ }\readlist\vars{#2}% read list of arguments separated by space
	\foreachitem\var\in\vars{\var\;}\mspace{-5mu}% print out all arguments separated by \;, and undoing the trailing \; by an mspace
	\IfValueT{#3}{\colon #3}%
	.\ %
}

% (partially copied from frmacros, Florian Rabe's personal collection of LaTeX macros)

\newcommand{\mmt}{\texorpdfstring{{\normalfont\scshape{Mmt}}\xspace}{MMT\ }}
\newcommand{\omdoc}{{\scshape{OMDoc}}\xspace}
\newcommand{\omdocmmt}{{\scshape{OMDoc}}/\mmt}
\newcommand{\mmtlf}{\texorpdfstring{{\normalfont\scshape{Mmt}/LF}\xspace}{MMT/LF\ }}
\newcommand{\ommt}{\omdocmmt}
\newcommand{\mathml}{{\scshape{MathML}}\xspace}
\newcommand{\openmath}{{\scshape{OpenMath}}\xspace}

\newcommand{\keyword}[1]{\operatorname{\mathbf{#1}}}

\newcommand{\mor}{\ensuremath{\rightsquigarrow}}
\newcommand{\includedIn}{\ensuremath{\hookrightarrow}}
\newcommand{\including}{\ensuremath{\hookleftarrow}}

\newcommand{\id}[1]{\mathrm{id}_{#1}}
% ==================================================

% LF
% ==================================================
% LF
% ==================================================
% incl. macros for typesetting variadic function
% both simply- and dependently-typed, variadic
% functions; all in different variants
% ==================================================

\makecn{type}
% judgements of LF
\newcommand\LFded{\vdash}

% anonymous functions, see \quantifier
\NewDocumentCommand\lam{m o}{\quantifier{\lambda}{#1}[#2]}
\newcommand{\ignorelam}{\lam{\_}}% for arguments thrown away

% see \quantifier
% useful in cases where one talks abount bound variables irrespective whether they are bound by lambda or Pi
\NewDocumentCommand\lampi{m o}{\quantifier{\lambda\kern -0.1em\Pi\,}{#1}[#2]}

% dependent function types, see \quantifier
\NewDocumentCommand\tpi{m o}{\quantifier{\Pi\,}{#1}[#2]}

% Typeset #1 -> ... -> #2 -> #3 with thin spacing between arrows and in ...
\newcommand{\flexFunType}[3]{\ensuremath{#1\denseTo\denseLdots\denseTo#2\denseTo#3}}

% Typeset #1 -> ... -> #2 with thin spacing between arrows and in ...
\newcommand{\flexHomogeneousFunType}[2]{\ensuremath{#1\denseTo\denseLdots\denseTo#2}}

% Typeset \lambda #1:#2 ... #3:#4 where the types can be given
% as optional arguments.
% E.g. `\flexFun{s}[S]{t}[T]` typesets `\lambda s: S. ... \lambda t: T.`
% And `\flexFun{s}{t}` typesets just `\lambda s. ... \lambda t.`.
\NewDocumentCommand\flexFun{m o m o}{\ensuremath{%
		\lambda\,#1\IfValueT{#2}{\kern -0.1em\colon\kern -0.08em#2}.\,%
		\denseLdots\kern 0.1em%
		\lambda\,#3\IfValueT{#4}{\kern -0.1em\colon\kern -0.08em#4}.%
		\hspace{0.6em}%
}}

% Typeset `\lambda #1:#2 \lambda #3:#4 ... \lambda #5:#6 \lambda #7:8` where the types can be given
% as optional arguments.
% E.g. `\flexFun{x_1}[X_1]{x_1'}[X_1']{x_n}[X_n]{x_n'}[U]` typesets `\lambda x_1:X_1 \lambda x_1':X_1' ... \lambda x_n:X_n \lambda x_n':X_n'`
%
% useful in the setting of logical relations along one morphisms (each pair of lambdas binds a term and a proof of it being in the relation)
\NewDocumentCommand\flexDoubleFun{momo momo}{\ensuremath{%
		\lambda\,#1\IfValueT{#2}{\kern -0.1em\colon\kern -0.08em#2}.\,%
		\lambda\,#3\IfValueT{#4}{\kern -0.1em\colon\kern -0.08em#4}.\,%
		\denseLdots\kern 0.1em%
		\lambda\,#5\IfValueT{#6}{\kern -0.1em\colon\kern -0.08em#6}.%
		\lambda\,#7\IfValueT{#8}{\kern -0.1em\colon\kern -0.08em#8}.%
		\hspace{0.6em}%
}}

% Use like:
% \flexTripleFunStart{x_1}{x_1'}{x_1^\ast}\flexTripleFunEnd{x_n}{x_n'}{x_n^\ast}
% useful in the setting of logical relations along two morphisms (each triple of lambdas binds two terms and a proof of them being in the relation)
%
% Split into "Start" and "End" commands because LaTeX cannot accept >= 10 parameters.
\NewDocumentCommand\flexTripleFunStart{momomo}{\ensuremath{%
		\lambda\,#1\IfValueT{#2}{\kern -0.1em\colon\kern -0.08em#2}.\,%
		\lambda\,#3\IfValueT{#4}{\kern -0.1em\colon\kern -0.08em#4}.\,%
		\lambda\,#5\IfValueT{#6}{\kern -0.1em\colon\kern -0.08em#6}.\,%
		\denseLdots\kern 0.1em%
}}
\NewDocumentCommand\flexTripleFunEnd{momomo}{\ensuremath{%
		\lambda\,#1\IfValueT{#2}{\kern -0.1em\colon\kern -0.08em#2}.%
		\lambda\,#3\IfValueT{#4}{\kern -0.1em\colon\kern -0.08em#4}.%
		\lambda\,#5\IfValueT{#6}{\kern -0.1em\colon\kern -0.08em#6}.%
		\hspace{0.6em}%
}}

% typeset `\Pi #1 #2 #3. ... \Pi #4 #5 #6.`
% all Pis are untyped
\NewDocumentCommand\flexTripleDepFun{mmmmmm}{\ensuremath{%
		\tpi{#1\,#2\,#3}\denseLdots\kern 0.1em\tpi{#4\,#5\,#6}%
}}

% Typeset "\Pi #1:#2 ... \Pi #3:#4." with good spacing and dense ldots
\NewDocumentCommand\flexPi{momo}{\ensuremath{%
		\tpi{#1}[#2]\,\denseLdots\,\tpi{#3}[#4]%
}}

% Typeset #1 #2 ... #3
\newcommand{\flexApp}[3]{\ensuremath{#1\ #2\ \denseLdots\ #3}}

% Typeset "\Pi #1:#2 \Pi #3:#4 ... \Pi #5:#6 \Pi #7:#8." with good spacing and dense ldots
% If nineth argument is given, typesets
%   \Pi #1:#2 \Pi #3:#4
%   \vdots
%   \Pi #5:#6 \Pi #7:#8 #9
%
% useful for logical relations along a single morphism, there #3:#4 and #7:#8 represent the proof of #1:'2 and #5:#6 being in the relation, respectively
\NewDocumentCommand\flexDoublePi{momo momo o}{\ensuremath{%
		\IfValueTF{#9}{%
			\begin{aligned}[t]%
				&\tpi{#1}[#2]\tpi{#3}[#4]\\[-0.5em]%
				&\vdots\\[-0.4em]%
				&\tpi{#5}[#6]\tpi{#7}[#8]#9\\%
			\end{aligned}%
		}{%
			\tpi{#1}[#2]\tpi{#3}[#4]%
			\,\denseLdots\,%
			\tpi{#5}[#6]\tpi{#7}[#8]%
		}%
}}
\NewDocumentCommand\flexTriplePiStart{momomo}{%
	\begin{aligned}[t]%
		&\tpi{#1}[#2]\tpi{#3}[#4]\tpi{#5}[#6]\\[-0.5em]%
		&\vdots\\[-0.4em]%
	}
\NewDocumentCommand\flexTriplePiEnd{momomo o}{%
		&\tpi{#1}[#2]\tpi{#3}[#4]\tpi{#5}[#6]\IfValueT{#7}{#7}%
	\end{aligned}%
}
% ==================================================

% UNTYPED TERMS
% ==================================================
\makecn{term}
% ==================================================

% HARD-TYPED TERMS
% ==================================================
\makecn{tp}
\newcommand{\tm}[1]{\tmSymbol\ #1}
\newcommand\tmSymbol{\cn{tm}}
% ==================================================

% SOFT-TYPED TERMS
% ==================================================
\newcommand{\ofTermType}[2]{#1\mathrel{::}#2}
% ==================================================

% LOGIC (PL, FOL, SFOL, and more)
% ==================================================
% PROPOSITIONAL LOGIC + NATURAL DEDUCTION
% ==================================================
\makecn{PL}
\makecn{PLND}
\makecn{prop}

\RequirePackage{mathabx}
\newcommand\dedc{\cn{\mathnormal{\Vdash}}}
\newcommand\ded{\Vdash}

\newcommand\andc{\cn{\mathnormal{\wedge}}}

\newcommand\orc{\cn{\mathnormal{\vee}}}
\newcommand\orElim{\orc_{\cn{E}}}
\newcommand\orIntroL{\orc_{\cn{IL}}}
\newcommand\orIntroR{\orc_{\cn{IR}}}

% to typeset negation, use \neg (which is the same as \lnot as per LaTeX, see https://tex.stackexchange.com/questions/430677/what-is-the-difference-between-lnot-and-neg)

\newcommand\negc{\cn{\mathnormal{\lnot}}}
\NewDocumentCommand\negIntro{mo}{%
	\leftflowed{\cn{\negc I}_{#1\IfValueT{#2}{\colon #2}}}%
}
\newcommand\negElim{\midflowedcn{\negc E}}

\newcommand\impl{\Rightarrow}
\newcommand\implc{\mathnormal{\Rightarrow}}
\newcommand\biimpl{\Leftrightarrow}
\newcommand\biimplc{\mathnormal{\Leftrightarrow}}
\newcommand\biimplIntro{\leftflowedcn{\biimplc I}}
% elimination of the forward implication
\newcommand\biimplElimFwd{\midflowedcn{\biimplc \overset{{}_{\rightarrow}}{E}}}
% elimination of the backwrad implication
\newcommand\biimplElimBwd{\midflowedcn{\biimplc \overset{{}_{\leftarrow}}{E}}}

% latex already defines \equiv, so use \equivs (s for "symbol")
% deprecated, use \biimpl
\newcommand\equivs{\biimpl}
% deprecated, use \biimplc
\newcommand\equivc{\biimplc}
% deprecated, use \biimplIntro
\newcommand\equivIntro{\equivc_{\cn{I}}}
% deprecated, use \biimplElimL
\newcommand\equivElimL{\biimplElimL}
% deprecated, use \biimplElimR
\newcommand\equivElimR{\biimplElimR}

\newcommand\falsum{\cn{\mathnormal{\bot}}}
\newcommand\falsumIntro{\falsum_{\cn{I}}}

% Double-stroked not symbols
%
% Source: https://tex.stackexchange.com/a/41141/38074
% Author: egreg <https://tex.stackexchange.com/users/4427/egreg>
% License: CC BY-SA 3.0 <https://creativecommons.org/licenses/by-sa/3.0/>
\newcommand{\blnot}{\mathord{\mathpalette\xblnot\relax}}
\newcommand{\xblnot}[2]{
	\sbox2{$#1\lnot$}\vrule height 1.1\ht2 width0pt \ooalign{%
		\hphantom{$#1-$}\relax\cr
		\noalign{\vskip-.2\ht2}
		\hfil$#1\lnot$\hfil\cr
		\noalign{\vskip.4\ht2}
		\hfil$#1\lnot$\hfil\cr
		\noalign{\vskip-.2\ht2}}%
}
% END: Double-stroked not symbols
\newcommand\doublenegc{\cn{\mathnormal{\blnot}}}
\newcommand\doubleNegIntro{\doublenegc_{\cn{I}}}
% ==================================================


% UNTYPED AND TYPED (SORTED) FIRST-ORDER LOGIC
% ==================================================
\makecn{FOL}
% sorted (aka typed) FOL; polymorphic FOL; dependent FOL; polymorphic+dependent FOL
\makecn{SFOL}\makecn{TFOL}\makecn{PFOL}\makecn{DFOL}\makecn{PDFOL}
\makecn{tp}
\makecn{ind}
\newcommand\forallc{\cn{\mathnormal{\forall}}}

% use \doteq for \eq relation
% and \eqc for raw symbol
\newcommand\eqc{\cn{\mathnormal{\doteq}}}

% meta-level equality on SFOL types
\newcommand\tpeq{\mathrel{\overset{\ast}{=}}}
\newcommand\tpeqc{\mathnormal{\overset{\ast}{=}}}

\RequirePackage{esvect}
% transport value #1 along type equality #2
\NewDocumentCommand\tptrans{mm}{\vv{#1}^{#2}}

% forall (weird name to not clash with latex's \forall)
\NewDocumentCommand\fall{m o}{\quantifier{\forall\,}{#1}[#2]}

% exists (weird name to not clash with latex's \exists)
\NewDocumentCommand\xists{m o}{\quantifier{\exists\,}{#1}[#2]}

%\newcommand{\of}[2]{#1\,::\,#2}
%\newcommand{\denseOfTermType}[2]{#1\kern -0.1em::\kern -0.1em#2}

%\newcommand{\spacedtmS}[2]{\tmS{\kern 0.05em#1\kern 0.06em#2}}

% Typeset "\forall #1:#2, ..., #3:#4." with good spacing and dense ldots
% The types #2 and #4 are optional (and thus given via square brackets)
% Examples: `\flexForall{s}[S]{t}[T]`
%           `\flexForall{s}{t}`
\NewDocumentCommand\flexForall{m o m o}{\ensuremath{%
		\forall\, #1\IfValueT{#2}{\kern -0.1em\colon\kern -0.08em#2}%
		,\denseLdots,%
		#3\IfValueT{#4}{\kern -0.1em\colon\kern -0.08em#4}%
		.\hspace{0.6em}
}}

% Typeset "\forall #1:#2, ..., #3:#4." with dense spacing and dense ldots
%\newcommand{\denselexaryForall}[4]{\ensuremath{\forall\, #1\colon#2,\denseLdots,#3\colon#4.\hspace{0.6em}}}

\NewDocumentCommand\funSymType{mmm}{\flexFunType{\tm{#1}}{\tm{#2}}{\tm{#3}}}
\NewDocumentCommand\predSymType{mm}{\flexFunType{\tm{#1}}{\tm{#2}}{\prop}}

% Typeset dense "\colon\ded"
\newcommand\colonded{\colon\kern -0.25em\ded}
%% END: sorted first-order logic %%

%% NATURAL DEDUCTION FOR SORTED FIRST-ORDER LOGIC %%
\newcommand\forallElim{\midflowedcn{\mathnormal{\forall}E}}

% natural deduction kern preceeding other natural deduction macros below
\newcommand\ndleftkern{\kern0.3em}
% natural deduction following other natural deduction macros below
\newcommand\ndrightkern{\kern0.3em}

% "flowed"
\newcommand{\leftflowed}[1]{#1\ndrightkern}
\newcommand{\leftflowedcn}[1]{\leftflowed{\cn{#1}}}
\newcommand{\midflowed}[1]{\mathbin{\ndleftkern#1\ndrightkern}}
\newcommand{\midflowedcn}[1]{\midflowed{\cn{#1}}}
\newcommand{\rightflowed}[1]{\ndleftkern#1}
\newcommand{\rightflowedcn}[1]{\rightflowed{\cn{#1}}}

\NewDocumentCommand\forallIntro{mo}{%
	\leftflowed{\cn{\mathnormal{\forall}I}_{#1\IfValueT{#2}{#2}}}%
}
\NewDocumentCommand\existsElim{mm}{%
	\midflowed{\cn{\mathnormal{\exists}E}_{#1,#2}}%
}

\newcommand\frefl{\leftflowedcn{refl}}

\makecn{refl}
\makecn{symm}
\makecn{trans}

\newcommand\fexistsIntro{\leftflowedcn{existsI}}
\newcommand\existsIntro{\leftflowedcn{\exists I}}
\newcommand\fexistsElim{\midflowedcn{existsE}}

\newcommand\fimplIntro{\leftflowedcn{implI}}
\newcommand\fimplElim{\midflowedcn{implE}}

\newcommand\fandIntro{\leftflowedcn{andI}}
\newcommand\fandElimL{\rightflowedcn{andEL}}
\newcommand\fandElimR{\rightflowedcn{andER}}

\newcommand\forIntroL{\leftflowedcn{orIL}}
\newcommand\forIntroR{\leftflowedcn{orIR}}
\newcommand\forElim{\midflowedcn{orE}}

\newcommand\fforallIntro{\leftflowedcn{forallI}}
\newcommand\fforallElim{\midflowedcn{forallE}}

\newcommand\fdepimpl{\midflowedcn{depImpl}}
\newcommand\fdepand{\midflowedcn{depAnd}}

% use in \begin{align*} instead of \\ to issue a bigger break useful to separate things (e.g. inclusions and actual theory body)
\newcommand\alignskip{\\[0.6em]}
% If you want to typeset, e.g., "... forallE blub", use `\dotsBeforeFlow \fforallElim blub`.
% Do _not_ employ `\denseLdots \fforallElim blub` as this will miss suitable space before \denseLdots.
\newcommand\dotsBeforeFlow{\ndleftkern\denseLdots}
\makecn{cong}
\newcommand\fcong{\midflowedcn{cong}}

% typeset variable name "pf" (and variants) for LF variables representing proofs (in natural deduction calculi)
% \pf* typesets "pf'"
% \pf[x] typesets "{pf^{x}}" (thus it enables you to use `\lam{\pf[x]^p \pf[x]^s}`)
\NewDocumentCommand\pf{so}{%
	\IfValueTF{#2}{%
		{\mathit{pf^{#2}}}%
	}{%
		\IfBooleanTF{#1}{{\mathit{pf'}}}{\mathit{pf}}%
	}%
}

% for constant identifiers of axiom symbols
\newcommand\ax{\mathit{ax}}
%% END: natural deduction for sorted first-order logic %%
% ==================================================

% (DEPENDENT) CONJUNCTION
% ==================================================
\NewDocumentCommand\depandc{}{\andc_{-}}

\NewDocumentCommand\depand{mo}{%
	\ndleftkern\wedge_{#1\IfValueT{#2}{\colon #2}}\ndrightkern%
}
% alias for \depand
\NewDocumentCommand\depwedge{mo}{\depand{#1}[#2]}

\newcommand\andIntro{\leftflowedcn{\mathnormal{\wedge}I}}
\NewDocumentCommand\depandIntro{}{%
	\leftflowedcn{\mathnormal{\wedge}I}%
}

\newcommand\andElimL{\rightflowedcn{\mathnormal{\wedge}EL}}
\newcommand\andElimR{\rightflowedcn{\mathnormal{\wedge}ER}}
\NewDocumentCommand\depandElimL{}{\rightflowedcn{\mathnormal{\wedge}EL}}
\NewDocumentCommand\depandElimR{}{\rightflowedcn{\mathnormal{\wedge}ER}}
% ==================================================

% (DEPENDENT) IMPLICATION
% ==================================================
\NewDocumentCommand\depimplc{}{\implc_{-}}
\NewDocumentCommand\depimpl{mo}{\midflowed{\implc_{#1\IfValueT{#2}{\colon #2}}}}

\NewDocumentCommand\implIntro{mo}{%
	\leftflowed{\cn{\implc I}_{#1\IfValueT{#2}{\colon #2}}}%
}
\NewDocumentCommand\depimplIntro{mo}{%
	\leftflowed{\cn{\implc I}_{#1\IfValueT{#2}{\colon #2}}}%
}

\NewDocumentCommand\implElim{}{%
	\midflowedcn{\implc E}%
}
\NewDocumentCommand\depimplElim{}{%
	\midflowedcn{\implc E}%
}
% ==================================================

% SORT OPERATORS
% ==================================================

% for predicate subtypes
% \usepackage[llbrace,rrbrace]{stmaryrd} --> option clash?
% ideally, we would use llbrace and rrbrace from stmaryrd
\NewDocumentCommand\subtype{m m}{%
	\lVert #1 \kern-0.35em\mid\kern-0.35em #2 \rVert%
}
\NewDocumentCommand\downcast{m m}{#1 \kern-0.2em\downarrow\kern-0.2em #2}
\NewDocumentCommand\upcast{m}{#1\kern-0.1em\uparrow}
\NewDocumentCommand\proofcast{m}{#1 ?}
% ==================================================
% ==================================================